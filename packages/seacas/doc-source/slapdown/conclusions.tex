\chapter{Conclusions}
\begin{enumerate}
\item {A simple three degree-of-freedom code, \SLAP ,
has been written to approximate the eccentric impact response
of a deformable body.
Nonlinear load
displacement characteristics and friction effects are included.  The
code has been verified experimentally and analytically.  The code
interfaces with Department
1520 plotting codes to provide convenient
graphical output.}

\item {The secondary impact velocity of a body
can be conveniently estimated using only the length
and radius of gyration.  Slapdown (velocity at secondary impact higher
than the primary impact velocity) cannot occur for length to radius of
gyration ratios less than two.}

\item {The amount of energy absorbed in the initial impact is the most
important parameter associated with the nose spring characteristics.
For linear elastic springs, the
spring rate (stiffness) of the nose spring is unimportant.}

\item {Friction, for geometries and coefficients reasonably associated
with transportation casks, has a small effect on secondary impact
velocity.  There is an optimum value (one which minimizes the
secondary impact velocity) of coefficient of friction based on the
load displacement characteristics of the nose spring and on the object
geometry.  Sufficient
friction can increase the severity of the
primary impact
to values greater than those
experienced for the flat side impact.  This can make the
primary impact at a shallow angle the controlling impact event.}

\item {The following scaling parameters have been verified for nonlinear
as well as linear load displacement characteristics (one G field
neglected):}
\end{enumerate}
\begin{table}
\begin{center}
\caption{Summary of Relationships for Scale Model Testing}
\makeqnum
\begin{tabular}{||l|c||}
\hline
\multicolumn{1}{|c}{Parameter}
 &\multicolumn{1}{|c||}{Scaling Relationships}\\
Geometry and & \\
Initial Conditions: &\\
\quad Overall Length     & $l_{sm} = l_{fs} \times (Scale)^{1}$\\
\quad Mass               & $M_{sm} = M_{fs} \times (Scale)^{3}$\\
\quad Moment of Inertia  & $I_{sm} = I_{fs} \times (Scale)^{5}$\\
\quad Spring Constants   & $K_{sm} = K_{fs} \times (Scale)^{1}$\\
\quad Initial Velocity   & $V_{sm} = V_{fs} \times (Scale)^{0}$\\
\quad Initial Angle      & $\theta _{sm} = \theta _{fs} \times
(Scale)^{0}$\\
\hline
Results: & \\
\quad    Linear Accelerations  & $a_{sm} = a_{fs} \times (Scale)^{-1}$\\
\quad    Angular Accelerations & $\alpha _{sm} = \alpha _{fs} \times
(Scale)^{-2}$\\
\quad    Linear Velocities     & $V_{sm} = V_{fs} \times (Scale)^{0}$\\
\quad    Angular Velocities    & $\omega _{sm} = \omega _{fs} \times
(Scale)^{-1}$\\
\quad    Linear Displacements  & $\Delta _{sm} = \Delta _{fs} \times
(Scale)^{1}$\\
\quad    Angular Displacements & $\theta _{sm} = \theta _{f} \times
(Scale)^{0}$\\
\hline
\end{tabular}
\end{center}
\end{table}
