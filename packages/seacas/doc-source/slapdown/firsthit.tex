\chapter{Effect of Initial Impact}

\section{Introduction}

     The behavior of the initial contact point is critical to the
behavior of the subsequent slapdown event.  In this section the effect
of various of load-displacement curve shapes (nonlinear springs) and
various amounts of energy absorption (low or high unloading modulus)
are investigated.  As in the previous section, the solid cylinder of
length 120 and radius 30 is used.  The analysis was performed for an
initial vertical velocity of -527.5 and an initial angle of 15$^\circ$.
The 15$^\circ$ initial angle was chosen to ensure that the initial nose
impact was completely over (forces on the nose were zero) prior to
contact of the tail spring.  Thus for the cases of elastic nose
springs, the nose rebound was complete and all recoverable energy
had been retransmitted to the cylinder. The model, at a total mass of
80, has a moment of inertia of 114,000, a radius of gyration of 37.75,
and a slenderness ratio of 3.18. The secondary impact severity is
represented by tail velocity at impact and by maximum tail
displacement.  As before, a linear tail spring is used so that tail
displacement is a measure of the energy the tail is required to
absorb.

\section{Nose Spring Definition}

     Seven nose springs were analysed.  Three of the springs were
linear elastic with spring rates that varied from soft (150,000) to
moderate (600,000) to almost rigid (600,000,000,000).  Two linear
plastic springs were also used.  The loading spring rate match the
soft and moderate linear elastic springs but the unloading rate was
very stiff (600,000,000,000).  This resulted in a spring in which the
loads were proportional to the displacement for comparison to the
elastic springs but in which all the energy was absorbed in the spring
(there was no rebound).  Finally, two nonlinear springs were analysed.
These springs had an initial spring rate of 400,000 for a unit
displacement.  After the unit displacement, the spring force was held
constant at 400,000 for all subsequent displacement.  In one case the
spring unloaded along the initial 400,000 spring rate recovering a
small portion of the stored energy.  In the other case the unloading
spring rate was 600,000,000,000 resulting in no significant energy
recovery and thus no rebound.  The load-displacement curves for these
seven springs are shown in Figures 5.1-5.3.

\begin{figure}
\vspace{3.5 in}
\caption{Load versus Displacement (Spring Rate) Curves for the Linear
Elastic Springs}
\end{figure}

\begin{figure}
\vspace{3.5 in}
\caption{Load versus Displacement (Spring Rate) Curves for the Linear
Plastic Springs}
\end{figure}

\begin{figure}
\vspace{3.5 in}
\caption{Load versus Displacement (Spring Rate) Curves for the
Nonlinear Springs}
\end{figure}

\section{Conclusions}

     The results of the slapdown analysis for the
solid cylinder with these
seven nose spring characteristics are presented in Table 5.1.  As can
be seen in Table 5.1, energy absorption is the only characteristic of
the nose spring which significantly affects the secondary impact
severity.  This confirms the result of Equation 2.3.18 on Page 16,
where it was shown that, for linear elastic springs, the tail impact
velocity is
a function of geometry only, and not of nose spring rate.

\begin{table}
\begin{center}
\caption{Effect of Initial Impact Resilience on Secondary Impact}
\begin{tabular}{||l|l|l|l|l|l||}
\hline
\multicolumn{1}{||c}{Nose Spring}
&\multicolumn{1}{|c}{Nose Spring}
&\multicolumn{1}{|c}{Nose Spring}
&\multicolumn{1}{|c}{Nose Spring}
&\multicolumn{1}{|c}{Tail}
&\multicolumn{1}{|c||}{Tail}\\
\multicolumn{1}{||c}{Type}
&\multicolumn{1}{|c}{Displ}
&\multicolumn{1}{|c}{Energy}
&\multicolumn{1}{|c}{Energy Absorbed}
&\multicolumn{1}{|c}{Vel}
&\multicolumn{1}{|c||}{Displ}\\
\hline
Soft L-E &$6.628$ &$3.295\times10^6$ &$0.$ &$-982$ &$6.035$\\
Med L-E  &$3.318$ &$3.303\times10^6$ &$0.$ &$-979$ &$6.029$\\
Rigid    &$0.003326$ &$3.319\times10^6$ &$0.$ &$-972$ &$6.018$\\
Soft L-P &$6.637$ &$3.304\times10^6$ &$3.304\times10^6$ &$-753$ &$5.073$\\
Med L-P  &$3.326$ &$3.319\times10^6$ &$3.319\times10^6$ &$-754$ &$4.993$\\
Nonlin-E &$8.699$ &$3.280\times10^6$ &$3.080\times10^6$ &$-808$ &$4.972$\\
Nonlin-P &$8.707$ &$3.283\times10^6$ &$3.283\times10^6$ &$-751$ &$5.102$\\
\hline
\end{tabular}
\end{center}
\end{table}

This conclusion must be modified for
application to actual transportation systems.  First,
most impact limiting systems used for
transportation systems
are relatively symmetric.  Therefore, a softer
nose spring implies a softer tail spring.  Thus,
while the tail velocity and tail energy absorption do not
change, the tail accelerations will be lower than for harder
springs.  The second effect of the use of softer nose springs results
from the larger displacements required to stop the nose.  These larger
nose displacements in turn result in the requirement of a larger
initial angle to ensure that the initial nose impact is complete prior
to tail contact.  Provided the initial nose impact is complete, the
severity of the tail impact will decrease with increasing
initial angle.
