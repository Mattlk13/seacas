\documentstyle[12pt]{smemo}
\def\SLAP{{\sf Slap\-down}}
\def\slap{{\sf Slap\-down}}
\def\exo{{\sf EXODUS}}
\newcommand{\caps}[1]
   {\uppercase{#1}\null}
\newcommand{\cmd}[1]
   {\mbox{\sf\uppercase{#1}}\null}
\newcommand{\lcmd}[1]
   {\mbox{\sf#1}\null}
\newcommand{\param}[1]
   {{\em #1}\null}
\newcommand{\optparam}[1]
   {[{\em #1\/}]\null}
\newcommand{\bold}[1]
   {{\bf #1}\null}
%%%
% STANDARD SETUP FOR MANUALS
%
%%% set up so item does not have so much space between items and top
\setlength{\parindent}{0pt}
\setlength{\topsep}{0pt} \setlength{\parsep}{0pt}
%%% \setlength{\itemsep}{\smallskipamount} \begin{itemize} ...
%
%%% \myindent is a nice amount to indent
\newlength{\myindent}
\setlength{\myindent}{30pt}
\newlength{\myhalfind}
\setlength{\myhalfind}{15pt}
%
%%% \cenlinesbegin and end start a block of centered text (each line to left)
\newcommand{\cenlinesbegin}
   { \begin{center} \begin{tabular}{l} }
\newcommand{\cenlinesend}
   { \end{tabular} \end{center} }
%
%%% these commands are used for all-capital-letter words (\caps),
%%% typed literals (\cmd), parameters (\param), bold font (\bold)
\newcommand{\caps}[1]
   {\uppercase{#1}\null}
\newcommand{\cmd}[1]
   {\mbox{\sf\uppercase{#1}}\null}
\newcommand{\lcmd}[1]
   {\mbox{\sf#1}\null}
\newcommand{\param}[1]
   {{\em #1}\null}
\newcommand{\optparam}[1]
   {[{\em #1\/}]\null}
%
%%% \negmedskip produces a medium negative space
\newcommand{\negmedskip}
   {\vspace{-\medskipamount}}
%
%%% \doublehline[#columns] produces a double horizontal line in tabular
\newcommand{\doublehline}[1]
   {\hline\multicolumn{#1}{c}{\vspace{-.17in}}\\ \hline}
%
%%% \SNL produces the text "Sandia National Laboratories"
%%% \SNLA adds "Albuquerque, NM"
\def\SNL{Sandia National Laboratories}
\def\SNLA{Sandia National Laboratories, Albuquerque, NM}
%%%
%%% \nth produces text like 1st with the "st" raised
\newcommand{\nth}[2]
   {{#1}\raisebox{.6ex}{#2}}
%
%%% \cmdverb is used to produce the command verb
\newfont{\bldsf}{cmssbx10 scaled \magstep1} % bold sans serif
\newcommand{\cmdverb}[1]{\mbox{\bldsf #1}\null}
%
%%% \default produces the command parameter default in braces;
%%% \nodefault is used for a command parameter with no default
\newcommand{\default}[1]{$<${#1}$>$}
\newcommand{\nodefault}[0]{$<$no default$>$}
%
%%% \topicbegin and end indents a block of text
\newcommand{\topicbegin}{
   \par\addtolength{\leftskip}{\myindent}
   \addtolength{\leftmargin}{\myindent}
   \addtolength{\leftmargini}{\myindent}
   }
\newcommand{\topicend}{
   \par\addtolength{\leftskip}{-\myindent}
   \addtolength{\leftmargin}{-\myindent}
   \addtolength{\leftmargini}{-\myindent}
   }
%
%%% \cmddef produces the command descriptions
\newcommand{\cmddef}[2]{
   \par \hangindent=\myhalfind \hangafter=1 {#1}
   \par \topicbegin {#2} \topicend
   }
%
%%% \cmdnext precedes a second command definition line
\newcommand{\cmdnext}{
   \par \hangindent=\myhalfind \hangafter=1 \negmedskip
   }
% \newcommand{\cmdnext}{
%   \par \hangindent=\myhalfind \hangafter=1 \vspace{-\bigskipamount}
%   }
%
%%% \cmdoption produces a heading and an indented block of text
\newcommand{\cmdoption}[2]{
   \par \hangindent=\myhalfind \hangafter=1 {#1}
   \par \vspace{-\medskipamount}
   \topicbegin {#2} \topicend
   }
%
%%% \cmdsum produces a command summary line
\newcommand{\cmdsum}[2]{
   \par {#1}
   \par \vspace{-\medskipamount}
   \topicbegin {#2} \topicend
   }
%
%%% \errfmt produces a centered error message
\newcommand{\errfmt}[1]{
   \begin{center} \vspace{-\smallskipamount}
   {#1}
   \vspace{-\smallskipamount} \end{center}
   }
%
%%% \notetome produces a message to me
\newcommand{\notetome}[1]{
%%%   \begin{center} {\bf *** {#1} ***} \end{center}
   \typeout{*** #1 ***}
   }
%
%%% The following are symbols set up for the names of the manuals.
%%% These symbols are used to check for equality.
\newcommand{\ABAEXO}{ABAEXO}
\newcommand{\ALGEBRA}{ALGEBRA}
\newcommand{\BLOT}{\sf Blot}
\newcommand{\GENTD}{GENTD}
\newcommand{\GJOIN}{GJOIN}
\newcommand{\GROPE}{GROPE}
\newcommand{\NUMBERS}{NUMBERS}

\newcommand{\superscript}[1]{\ensuremath{^{\textrm{#1}}}}
\newcommand{\subscript}[1]{\ensuremath{_{\textrm{#1}}}}
% Welcome to the 21\superscript{st} century.
\renewcommand{\th}[0]{\superscript{th}}
\newcommand{\st}[0]{\superscript{st}}
\newcommand{\nd}[0]{\superscript{nd}}
\newcommand{\rd}[0]{\superscript{rd}}
% Carl took 1\st place. Elly, Tom and Ann came in 2\nd, 3\rd and 4\th.

\newcommand{\texifylineno}[1]{{\footnotesize\hspace{-4em}\makebox[4em][r]{#1\hspace{1.5ex}}}}

\newcommand{\sectionline}{%
  \nointerlineskip \vspace{\baselineskip}%
  \hspace{\fill}\rule{\linewidth}{.7pt}\hspace{\fill}%
	\par\nointerlineskip
}
\def\SNLA{Sandia National Laboratories, Albuquerque, NM}

\newcommand\dash{\raise2.1pt\hbox{\rule{5pt}{0.3pt}}\hspace{1pt}}
\newcommand{\gap}{\hspace{12pt}}

\newcommand{\paramdef}[2]
 {\color{red}\begin{tabular}{@{}l@{}p{6.0in}}\bf\tt #1\enspace\enspace &\tt #2\\\index{#1!Parameter}\index{Parameter!#1}\end{tabular}\color{black}}

\newcommand{\funcdef}[2]
 {\color{red}\begin{tabular}{@{}l@{}p{0.5in}}\Large\bf\tt int #1\enspace(&\Large\bf\tt #2)\\\index{#1!Definition}\index{Functions!#1!Definition}\end{tabular}\color{black}}
\newcommand{\funcdefv}[2]
 {\color{red}\begin{tabular}{@{}l@{}p{3.7in}}\Large\bf\tt void #1\enspace(&\Large\bf\tt #2)\\\index{#1!Definition}\index{Functions!#1!Definition}\end{tabular}\color{black}}
%
% Set left margin and counter
\setlength{\oddsidemargin}{0.25in}
\setlength{\evensidemargin}{0.25in}
\setcounter{secnumdepth}{10}
\setcounter{tocdepth}{3}
\setlength{\textwidth}{6in}
\setlength{\topmargin}{-0.25in}
\setlength{\textheight}{8.5in}
\parskip 0.2cm
\parindent 0.0cm

\newenvironment{syntax}
{\begin{quote}\begin{alltt}\color{blue}}
{\end{alltt}\end{quote}}

\newenvironment{csource}
{\small\begin{quote}\color{blue}\begin{alltt}}
{\end{alltt}\end{quote}\normalsize}

\newenvironment{fsource}
{\small\begin{quote}\color{red}\begin{alltt}}
{\end{alltt}\end{quote}\normalsize}

\newcommand{\api}
   {\mbox{\em API Functions:\hspace{0.1in}}\null}

\newcommand{\exo}
   {\mbox{\sf Exodus}\null}

\newcommand{\exodiff}
   {\mbox{\sf Exodiff}\null}

\newcommand{\epu}
   {\mbox{\sf EPU}\null}

\newcommand{\ejoin}
   {\mbox{\sf EJoin}\null}

\newcommand{\conjoin}
   {\mbox{\sf Conjoin}\null}

\newcommand{\entrylabel}[1]{
   {\parbox[b]{\labelwidth-4pt}{\makebox[0pt][l]{\color{red}\tt\textbf{#1}}\vspace{1.0\baselineskip}}}}
\newenvironment{parameters}
{\begin{list}{}
  {
    \settowidth{\labelwidth}{80pt}
    \setlength{\leftmargin}{\labelwidth}
    \setlength{\parsep}{-10pt}
    \setlength{\itemsep}{24pt}  % Between items
    \renewcommand{\makelabel}{\entrylabel}
  }
}
{\end{list}}

\begin{document}
\begin{memo}

\to{ Distribution}
\from{G. D. Sjaardema, 1521}
\subject Modifications to \slap

\SLAP\ is a computer program written to help analysts determine the
behavior of a body impacting onto an unyielding surface.  The program
models the body as a three degree-of-freedom system.  The deformation of
the body is approximated by nonlinear springs at each end of the model.
The program was written to model the impact behavior of both nuclear
waste transportation casks (subjected to the regulatory 30-foot drops
onto an unyielding surface) and laydown weapons.

The code documentation and users' manual are in SAND88-0616, "Numerical
and Analytical Methods for Approximating the Eccentric Impact Response
(Slapdown) of Deformable Bodies."  However, several modifications and
improvements have been added to \slap\ since the users' manual was
written.  The modifications and improvements are documented in this
memo.

\section*{New Commands}

A \slap\ analysis normally continues until either both springs have
unloaded, or the velocities of both the \cmd{NOSE} and \cmd{TAIL} are
positive.  Sometimes a longer or shorter time duration is needed for
special cases; therefore, a command has been added to let the analyst
control the length of the analysis.  The command syntax is:
\begin{quote}
\sf TERMINATION [TIME] \em end\_time
\end{quote}
where the keyword \cmd{TIME} is optional, and the parameter
\param{end\_time} is the time to terminate the analysis.

If \cmd{TERMINATION~TIME} is not specified, the calculation will
continue until either both springs have unloaded, or the velocities of
both the \cmd{NOSE} and \cmd{TAIL} are positive.

The square end treatment (see Section~3.1, page~19 of the users' manual)
can now be specified for both the nose and tail of the body; previously
it could only be specified for the nose.  The command syntax is:
\begin{quote}
\sf TAIL SQUARE \\
\end{quote}
\rm or, if the nose has been specified as \cmd{SQUARE}, the \cmd{TAIL
SYMMETRIC} command will also result in a square tail.

\section*{Friction Algorithm Improved}

The friction algorithm has been extensively modified and improved.  The
algorithm correctly handles both square and round ends and will
``stick'' at the impact point if the coefficient of friction is high
enough.  The new algorithm also works when both ends are in contact
simultaneously--the oscillations that occurred with the old algorithm no
longer occur.

The algorithm uses an iterative approach to determine the lateral force
required to completely stick the impacting end.  It then compares this
with the maximum possible lateral force (normal force times the
coefficient of friction) and returns the minimum of these two values. In
several test problems and a comparison with an actual test, the new
algorithm gives much better results.

\section*{Modifications to \slap\ Output Files}

\paragraph*{\exo\ Database Changes:} Two new nodes have been added to
the \exo\ database file.  The left end contact point is node~4 and the
right end contact point is node 5. Figure~\ref{geometry} shows the node
and element numberings used in \slap.  The nodes at the contact points
were added to provide information about the behavior at these points for
bodies with square ends. For square ends, the $x$-velocities at the
contact points are not the same as the $x$-velocities at the ends of the
body centerline.

The global variable \cmd{DAMKE} (damaging kinetic energy) has been added
to the database.  This variable is sometimes used in laydown weapon
analysis to indicate the severity of the impact.  The damaging kinetic
energy is equal to
\begin{displaymath}
\cmd{DAMKE} = \sfrac1/2 \left(m v_y^2 + I \omega^2\right)
\end{displaymath}
where $m$ is the mass, $I$ is the mass moment of inertia, $v_y$ is the
vertical velocity at the center of gravity, and $\omega$ is the angular
velocity about the center of gravity.

\paragraph*{Output File Changes:}  The text output file has also been
modified slightly.  The event sequence summary (see Figure~A.5 in the
users' manual) now includes the ratio of the impact velocity to the
initial velocity which makes it easier to determine the severity of the
slapdown effect.

Also, the maximum strain energy in each of the springs during the impact
event is output in the results summary.  These values are important
in determining the severity of the slapdown effect since for some
geometries the second impact will occur at a velocity greater than the
initial velocity; however, due to the bodies angular position and
velocity, the strain energy in the spring will be less than that caused
by a flat impact.  The opposite effect will also occur (higher strain
energy at lower impact velocity).

\section*{Support and Notification of Modifications}
An \cmd{ICER} facility has been created for \slap.  This facility will
be used to provide a more immediate notification of any future
modifications or problems. If there are any problems with \slap, contact
Greg Sjaardema, 1521, at 844--7045.

\begin{figure}
%\begin{center}
\unitlength 1in
\begin{picture}(6.0,4)
\thicklines
\put(0.25, 2.5){\framebox(.75,.3){Nose}}
\put(5.00, 3.5){\framebox(.75,.3){Tail}}
\put(3.125,2.0){\framebox(.75,.3){CG}}
\put(1.25, 1){\line(-1,4){0.25}}
\put(1.00, 2){\line( 4,1){4.0}}
\put(5.00, 3){\line(0,-1){1.0}}
\put(3.50, 2.6250){\circle*{.1}}
%\put(5.25, 2.4000){\vector(0,-1){.4}}

\put(0.8, 2.0000){\makebox(0,0){N1}}
\put(3.50, 2.7750){\makebox(0,0){N2}}
\put(5.2, 3.0000){\makebox(0,0){N3}}
\put(1.4, 1.1){\makebox(0,0){N4}}
\put(5.2, 1.9500){\makebox(0,0){N5}}
\put(2.25, 2.4625){\makebox(0,0){E1}}
\put(4.25, 2.9625){\makebox(0,0){E2}}
%\put(5.25, 2.5000){\makebox(0,0){$R_t$}}
%\put(5.25, 2.6000){\vector(0, 1){.4}}
%
% force vectors
%
%\put(3.50, 2.6250){\vector(0, 1){.5}}
%\put(3.50, 3.1250){\makebox(0,0)[b]{$V_{Y0}$}}
%\put(3.50, 2.6250){\vector(1, 0){.5}}
%\put(4.000,2.6250){\makebox(0,0)[l]{$V_{X0}$}}
%%
%\put(5.00, 1.5   ){\vector(0, 1){.5}}
%\put(5.00, 1.5   ){\makebox(0,0)[t]{$F_N^t$}}
%\put(4.50, 2.0   ){\vector(1, 0){.5}}
%\put(4.50, 2.0   ){\makebox(0,0)[r]{$F_T^t$}}
%%
%\put(1.25, 0.5   ){\vector(0, 1){.5}}
%\put(1.25, 0.5   ){\makebox(0,0)[t]{$F_N^n$}}
%\put(0.75, 1.0   ){\vector(1, 0){.5}}
%\put(0.75, 1.0   ){\makebox(0,0)[r]{$F_T^n$}}
%%
%% Dimensions
%%
%\put(1.40, 1.4){\vector(1,-4){.10}}
%\put(1.375,1.5){\makebox(0,0){$R_n$}}
%\put(1.35, 1.6){\vector(-1,4){.10}}
%%
%\put(2.10, 2.5250){\vector(-4,-1){1.10}}
%\put(2.25, 2.5625){\makebox(0,0){$L_n$}}
%\put(2.40, 2.6000){\vector( 4, 1){1.10}}
%%
%\put(4.10, 3.0250){\vector(-4,-1){0.60}}
%\put(4.25, 3.0625){\makebox(0,0){$L_t$}}
%\put(4.40, 3.1000){\vector( 4, 1){0.60}}
%
% Rigid surface
%
\put(1,1){\line(1,0){4.5}}
\put(3.5,.8){Rigid Surface}
%
\put(1.50,2){\line(1,0){0.5}}
\put(1.75,2.05){\makebox(0,0)[b]{$\theta$}}
\thinlines
%
\end{picture}
%\end{center}
\caption{\SLAP\ Node (N) and Element (E) Numberings ({\sf SQUARE}
end)}\label{geometry}
\end{figure}

\end{memo}
\end{document}
