\chapter{Introduction}

     There are many cases where the description of rigid body motion
during an impact event is desired.  Two examples are the response of a
radioactive materials transportation container to the regulatory
30-foot drop event and the effects of certain deployment options on
various weapon components. Rigid body motion effects are of primary
importance for eccentric impact orientations in which a vector normal
to the impact surface at the point of impact does not pass through the
center of gravity of the body.  In this impact orientation, the body
will tend to rotate about the impact point which will possibly result
in a secondary impact at the opposite end of the body.  This sequence
of a primary impact followed by a secondary impact at the opposite end
of the body is commonly called {\em slapdown}.  In many cases, the
secondary slapdown impact can occur at a higher velocity and be more
damaging than the primary impact. A computer program called \SLAP\ has
been written to approximate the slapdown behavior of deformable bodies
and to aid in the investigation of eccentric impact events.

In this report, the basic analytical approach to the problem of shallow angle
slapdown is shown in Chapter~2.  The numerical implementation of the solution
of the nonlinear equations of motion is developed in Chapter~3.  Applications
to a variety of generic slapdown issues are presented in Chapters~4--6.  These
generic issues include the effects of geometric variables, initial impact
behavior, and friction on the slapdown severity.  Chapter~7 details the proper
procedures to follow for extrapolation of scale model test results to the
full-scale response. A user's manual detailing the input required and the
output from \SLAP\ is presented in Appendix~A. The analytical results from
\SLAP\ are compared to the experimental results from a test program on a
half-scale model of an actual radioactive materials transportation container in
Appendix~B, and Appendix~C is a description of the \EXO\ database format.

The primary motivation for the development of this capability was to address
the behavior of radioactive materials transportation containers during shallow
angle slapdown impact events.  Therefore, the applications presented reflect
container geometries and impact limiter behavior typical of this field.  This
should not in any way be considered the exclusive use of the computer program.
