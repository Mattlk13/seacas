\chapter{Predefined Variables}\label{ch:predefined}

A few commonly used variables are predefined in
\aprepro{}\footnote{The units system described in
Chapter~\ref{ch:units} also predefines several variables when it is
activated}. These are listed below. The default output format
\cmd{\_FORMAT} is specified as a C language format string, see your C
language documentation for more information. The default output format
(\cmd{\_FORMAT}) and comment (\cmd{\_C\_}) variables are defined with
a leading underscore in their name so they can be redefined without
generating an error message.

\begin{longtable}{lll}
\caption{Predefined Variables}\\
Name & Value & Description \\
\hline
 PI      & 3.14159265358979323846 & $\pi$ \\
 PI\_2   & 1.57079632679489661923 & $\pi/2$ \\
 TAU     & 6.28318530717958623200 & $2\pi$ \\
 SQRT2   & 1.41421356237309504880 & $\sqrt{2}$ \\
 DEG     & 57.2957795130823208768 & $180/\pi$ degrees per radian \\
 RAD     & 0.01745329251994329576 & $\pi/180$ radians per degree \\
 E       & 2.71828182845904523536 & base of natural logarithm \\
 GAMMA   & 0.57721566490153286060 & $\gamma$, euler-mascheroni constant \\
 PHI     & 1.61803398874989484820 & golden ratio $(\sqrt{5}+1)/2$ \\
 TRUE    & 1 & \\
 FALSE   & 0 & \\
 \_VERSION\_ & Varies, string value   & current version of \aprepro \\
 \_FORMAT& \texttt{"\%.10g"} & default output format \\
  \_C\_  &  \texttt{"\$"} &  default comment character \\
\end{longtable}

Note that the output format is used to output both integers and
floating point numbers. Therefore, it should use the \texttt{\%g} format
descriptor which will use either the decimal (\texttt{\%d}), exponential (\texttt{\%e}),
or float (\texttt{\%f}) format, whichever is shorter, with insignificant zeros
suppressed.

If the output format is set to the empty string, the output will use
as many variables as needed to fully represent the double precision
value.  This can be selected at startup time using the command-line
option \textbf{--full\_precision} or \textbf{-p}.

The table below illustrates the effect of different format
specifications on the output of the variable \textbf{PI} and the value
1.0 . See the documentation of your C compiler for more
information. For most cases, the default value is sufficient.

If you need to temporarily change the output format for a specific
case, you can use the \cmd{format(var, format)} command which will
return a string representing \var{var} printed using the format
specified in \var{format}.
\begin{longtable}{lll}
\caption{Effect of various output format specifications}\\
\var{\_FORMAT} & PI Output & 1.0 Output \\
\hline
\%.10g &  3.141592654      & 1  \\
\%.10e &  3.1415926536e+00 & 1.0000000000e+00  \\
\%.10f &  3.1415926536     & 1.0000000000  \\
\%.10d &  1413754136       & 0000000000  \\
""  &  3.141592653589793 & 1 \\
\end{longtable}

The comment character should be set to the character that the program
which will read the processed file uses as a comment character. The
default value of \texttt{"\$"} is the comment character used
by the SEACAS codes at Sandia National Laboratories.  The \textbf{-c}
command line option (described in Chapter~\ref{ch:execution}) changes
the value of the comment variable to match the character specified on
the command line.
