%
% STANDARD SETUP FOR MANUALS
%
%%% set up so item does not have so much space between items and top
\setlength{\parindent}{0pt}
\setlength{\topsep}{0pt} \setlength{\parsep}{10pt}
%%% \setlength{\itemsep}{\smallskipamount} \begin{itemize} ...
%
%%% \myindent is a nice amount to indent
\newlength{\myindent}
\setlength{\myindent}{30pt}
\newlength{\myhalfind}
\setlength{\myhalfind}{15pt}
%
%%% \cenlinesbegin and end start a block of centered text (each line to left)
\newcommand{\cenlinesbegin}
   { \begin{center} \begin{tabular}{l} }
\newcommand{\cenlinesend}
   { \end{tabular} \end{center} }
%
%%% these commands are used for all-capital-letter words (\caps),
%%% typed literals (\cmd), parameters (\param), bold font (\bold)
\newcommand{\caps}[1]
   {\uppercase{#1}\null}
\newcommand{\cmd}[1]
   {\mbox{\sf\uppercase{#1}}\null}
\newcommand{\lcmd}[1]
   {\mbox{\sf#1}\null}
\newcommand{\param}[1]
   {{\em #1}\null}
\newcommand{\optparam}[1]
   {[{\em #1\/}]\null}
\newcommand{\bold}[1]
   {{\bf #1}\null}
\newcommand{\eref}[1]{{(\ref #1)\null}}
%
%%% \negmedskip produces a medium negative space
\newcommand{\negmedskip}
   {\vspace{-\medskipamount}}
%
%%% \doublehline[#columns] produces a double horizontal line in tabular
\newcommand{\doublehline}[1]
   {\hline\multicolumn{#1}{c}{\vspace{-.17in}}\\ \hline}
%
%%% \SNL produces the text "Sandia National Laboratories"
%%% \SNLA adds "Albuquerque, NM"
\def\SNL{Sandia National Laboratories}
\def\SNLA{Sandia National Laboratories, Albuquerque, NM}
%%%
%%% \nth produces text like 1st with the "st" raised
\newcommand{\nth}[2]
   {{#1}\raisebox{.6ex}{#2}}
%
%%% \cmdverb is used to produce the command verb
\newfont{\bldsf}{cmssbx10 scaled \magstep1} % bold sans serif
\newcommand{\cmdverb}[1]{\mbox{\bldsf #1}\null}
%
%%% \default produces the command parameter default in braces;
%%% \nodefault is used for a command parameter with no default
\newcommand{\default}[1]{$<${#1}$>$}
\newcommand{\nodefault}[0]{$<$no default$>$}
%
%%% \topicbegin and end indents a block of text
\newcommand{\topicbegin}{
   \par\addtolength{\leftskip}{\myindent}
   \addtolength{\leftmargin}{\myindent}
   \addtolength{\leftmargini}{\myindent}
   }
\newcommand{\topicend}{
   \par\addtolength{\leftskip}{-\myindent}
   \addtolength{\leftmargin}{-\myindent}
   \addtolength{\leftmargini}{-\myindent}
   }
%
%%% \cmddef produces the command descriptions
\newcommand{\cmddef}[2]{
   \par \hangindent=\myhalfind \hangafter=1 {#1}
   \par \topicbegin {#2} \topicend
   }
%
%%% \cmdnext precedes a second command definition line
\newcommand{\cmdnext}{
   \par\hangindent=\myhalfind \hangafter=1
   }
% \newcommand{\cmdnext}{
%   \par \hangindent=\myhalfind \hangafter=1 \vspace{-\bigskipamount}
%   }
%
%%% \cmdoption produces a heading and an indented block of text
\newcommand{\cmdoption}[2]{
   \par \hangindent=\myhalfind \hangafter=1 {#1}
   \topicbegin {#2} \topicend
   }
%
%%% \cmdsum produces a command summary line
\newcommand{\cmdsum}[2]{
   \par {#1}
   \par
   \topicbegin {#2} \topicend
   }
%
%%% \errfmt produces a centered error message
\newcommand{\errfmt}[1]{
   \begin{center}{#1}\end{center}
   }
%
%%% \notetome produces a message to me
\newcommand{\notetome}[1]{
%%%   \begin{center} {\bf *** {#1} ***} \end{center}
   \typeout{*** #1 ***}
   }
%
%%% The following are symbols set up for the names of the manuals.
%%% These symbols are used to check for equality.
\newcommand{\ABAEXO}{ABAEXO}
\newcommand{\ALGEBRA}{ALGEBRA}
\newcommand{\BLOT}{BLOT}
\newcommand{\GENTD}{GENTD}
\newcommand{\GJOIN}{GJOIN}
\newcommand{\EXPLORE}{EXPLORE}
\newcommand{\NUMBERS}{NUMBERS}
