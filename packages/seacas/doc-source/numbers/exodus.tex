\chapter{The EXODUS Database Format} \label{a:exodus}

The following code segment reads an EXODUS database. The first segment
is the GENESIS database format.

\begin{verbatim}
C   --Open the EXODUS database file

      NDB = 9
      OPEN (UNIT=NDB, ..., STATUS='OLD', FORM='UNFORMATTED')

C   --Read the title

      READ (NDB) TITLE
C      --TITLE - the title of the database (CHARACTER*80)

C   --Read the database sizing parameters

      READ (NDB) NUMNP, NDIM, NUMEL, NELBLK,
     &   NUMNPS, LNPSNL, NUMESS, LESSEL, LESSNL, NVERSN
C      --NUMNP - the number of nodes
C      --NDIM - the number of coordinates per node
C      --NUMEL - the number of elements
C      --NELBLK - the number of element blocks
C      --NUMNPS - the number of node sets
C      --LNPSNL - the length of the node sets node list
C      --NUMESS - the number of side sets
C      --LESSEL - the length of the side sets element list
C      --LESSNL - the length of the side sets node list
C      --NVERSN - the file format version number

C   --Read the nodal coordinates

      READ (NDB) ((CORD(INP,I), INP=1,NUMNP), I=1,NDIM)

C   --Read the element order map (each element must be listed once)

      READ (NDB) (MAPEL(IEL), IEL=1,NUMEL)
\end{verbatim}
\newpage
\begin{verbatim}
C   --Read the element blocks

      DO 100 IEB = 1, NELBLK

C      --Read the sizing parameters for this element block

         READ (NDB) IDELB, NUMELB, NUMLNK, NATRIB
C         --IDELB - the element block identification (must be unique)
C         --NUMELB - the number of elements in this block
C         --   (the sum of NUMELB for all blocks must equal NUMEL)
C         --NUMLNK - the number of nodes defining the connectivity
C         --   for an element in this block
C         --NATRIB - the number of element attributes for an element
C         --   in this block

C      --Read the connectivity for all elements in this block

         READ (NDB) ((LINK(J,I), J=1,NUMLNK, I=1,NUMELB)

C      --Read the attributes for all elements in this block

         READ (NDB) ((ATRIB(J,I), J=1,NATRIB, I=1,NUMELB)

  100 CONTINUE
\end{verbatim}
\newpage
\begin{verbatim}
C   --Read the node sets

      READ (NDB) (IDNPS(I), I=1,NUMNPS)
C      --IDNPS - the ID of each node set
      READ (NDB) (NNNPS(I), I=1,NUMNPS)
C      --NNNPS - the number of nodes in each node set
      READ (NDB) (IXNNPS(I), I=1,NUMNPS)
C      --IXNNPS - the index of the first node in each node set
C      --   (in LTNNPS and FACNPS)

      READ (NDB) (LTNNPS(I), I=1,LNPSNL)
C      --LTNNPS - the nodes in all the node sets
      READ (NDB) (FACNPS(I), I=1,LNPSNL)
C      --FACNPS - the factor for each node in LTNNPS

C   --Read the side sets

      READ (NDB) (IDESS(I), I=1,NUMESS)
C      --IDESS - the ID of each side set
      READ (NDB) (NEESS(I), I=1,NUMESS)
C      --NEESS - the number of elements in each side set
      READ (NDB) (NNESS(I), I=1,NUMESS)
C      --NNESS - the number of nodes in each side set
      READ (NDB) (IXEESS(I), I=1,NUMESS)
C      --IXEESS - the index of the first element in each side set
C      --   (in LTEESS)
      READ (NDB) (IXNESS(I), I=1,NUMESS)
C      --IXNESS - the index of the first node in each side set
C      --   (in LTNESS and FACESS)

      READ (NDB) (LTEESS(I), I=1,LESSEL)
C      --LTEESS - the elements in all the side sets
      READ (NDB) (LTNESS(I), I=1,LESSNL)
C      --LTNESS - the nodes in all the side sets
      READ (NDB) (FACESS(I), I=1,LESSNL)
C      --FACESS - the factor for each node in LTNESS
\end{verbatim}
\newpage

A valid GENESIS database may end at this point or at any point until the
number of variables is read.

\begin{verbatim}
C   --Read the QA header information

      READ (NDB, END=900) NQAREC
C      --NQAREC - the number of QA records (must be at least 1)

      DO 110 IQA = 1, MAX(1,NQAREC)
         READ (NDB) (QATITL(I,IQA), I=1,4)
C         --QATITL - the QA title records; each record contains:
C         --   1) analysis code name (CHARACTER*8)
C         --   2) analysis code qa descriptor (CHARACTER*8)
C         --   3) analysis date (CHARACTER*8)
C         --   4) analysis time (CHARACTER*8)
  110 CONTINUE

C   --Read the optional header text

      READ (NDB, END=900) NINFO
C      --NINFO - the number of information records

      DO 120 I = 1, NINFO
         READ (NDB) INFO(I)
C         --INFO - extra information records (optional) that contain
C         --   any supportive documentation that the analysis code
C         --   developer wishes (CHARACTER*80)
  120 CONTINUE

C   --Read the coordinate names

      READ (NDB, END=900) (NAMECO(I), I=1,NDIM)
C      --NAMECO - the coordinate names (CHARACTER*8)

C   --Read the element type names

      READ (NDB, END=900) (NAMELB(I), I=1,NELBLK)
C      --NAMELB - the element type names (CHARACTER*8)
\end{verbatim}

The GENESIS section of the database ends at this point.

\newpage
\begin{verbatim}
C   --Read the history, global, nodal, and element variable information

      READ (NDB, END=900) NVARHI, NVARGL, NVARNP, NVAREL
C      --NVARHI - the number of history variables
C      --NVARGL - the number of global variables
C      --NVARNP - the number of nodal variables
C      --NVAREL - the number of element variables

      READ (NDB)
     &   (NAMEHV(I), I=1,NVARHI),
     &   (NAMEGV(I), I=1,NVARGL),
     &   (NAMENV(I), I=1,NVARNP),
     &   (NAMEEV(I), I=1,NVAREL)
C      --NAMEHI - the history variable names (CHARACTER*8)
C      --NAMEGV - the global variable names (CHARACTER*8)
C      --NAMENV - the nodal variable names (CHARACTER*8)
C      --NAMEEV - the element variable names (CHARACTER*8)

      READ (NDB) ((ISEVOK(I,J), I=1,NVAREL), J=1,NELBLK)
C      --ISEVOK - the name truth table for the element blocks;
C      --   ISEVOK(i,j) refers to variable i of element block j;
C      --   the value is 0 if and only if data will NOT be output for
C      --   variable i for element block j (otherwise the value is 1)
\end{verbatim}
\newpage
\begin{verbatim}
C   --Read the time steps

  130 CONTINUE
         READ (NDB, END=900) TIME, HISTFL
C         --TIME - the time step value
C         --HISTFL - the time step type flag:
C         --   0.0 for all variables output ("whole" time step) else
C         --   only history variables output ("history-only" time step)

         READ (NDB) (VALHV(IVAR), IVAR=1,NVARGL)
C         --VALHV - the history values for the current time step

         IF (HISTFL .EQ. 0.0) THEN
            READ (NDB) (VALGV(IVAR), IVAR=1,NVARGL)
C            --VALGV - the global values for the current time step

            DO 140 IVAR = 1, NVARNP
               READ (NDB) (VALNV(INP,IVAR), INP=1,NUMNP)
C               --VALNV - the nodal variables at each node
C               --   for the current time step
  140       CONTINUE

            DO 160 IBLK = 1, NELBLK
               DO 150 IVAR = 1, NVAREL
                  IF (ISEVOK(IVAR,IBLK) .NE. 0) THEN
                     READ (NDB) (VALEV(IEL,IVAR,IBLK),
     &                  IEL=1,NUMELB(IBLK))
C                     --VALEV - the element variables at each element
C                     --   for the current time step
                  END IF
  150          CONTINUE
  160       CONTINUE
         END IF

C      --Handle time step data
      GOTO 130

  900 CONTINUE
C   --Handle end of file on database
\end{verbatim}
