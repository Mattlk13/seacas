\chapter{Introduction} \label{chap:intro}

\caps{\PROGRAM} translates the output of the \caps{ABAQUS} finite element
program \cite{bib:abaqus} into the \caps{EXODUS} database format
\cite{bib:exodus}. \caps{ABAEXO} translates the output of ABAQUS versions
4-5 and 4-6.
The \caps{EXODUS} format is used by the
Engineering Analysis Department plot programs such as \caps{BLOT}
\cite{bib:blot}. 

\caps{ABAQUS} allows 8-node 2D elements and 20-node 3D elements, but the
plot programs expect 4-node 2D elements and 8-node 3D elements.
\caps{\PROGRAM} transforms any 8-node 2D element into four 4-node
elements and any 20-node 3D element into eight 8-node elements. Any
multi-node 1D element is transformed into 2-node elements. There is also
an option to transform 4-node 2D elements into four 4-node elements.

\section{The Element Transformations} \label{intro:trans}

The 2D and 3D transformations are shown in Figures 1a and 1b. The solid
lines show the original elements. The dashed lines show the elements
that are generated. The circled nodes are nodes which are generated for
the transformation.

An 8-node 2D element is broken into four 4-node elements by adding a
center node. To transform a 4-node 2D element into four 4-node elements,
a node is added at the center of each side of the element, and then the
8-node transformation is performed. A side node generated by the 4-node
transformation may be shared by an adjacent element.

A 20-node 3D element is broken into eight 8-node elements by adding
seven nodes: one at the center of each side and one at the center of the
element. A generated center side node may be shared by an adjacent
element.

\newpage \addtocounter{page}{1}
\notetome{Insert figure produced by ABAFIG.EXE}

\newcommand{\nodenum}[1] {\em #1}

\section{Translation of Nodal Quantities} \label{intro:nodal}

Two types of information for each node are needed: the 
coordinates and the value of the nodal variables.

The \caps{ABAQUS} output file contains the coordinates of the 
original nodes.
The coordinates of the generated nodes
are the average of the coordinates of the original nodes that define the
new nodes. The center node of the 8-node 2D transformation (\nodenum{9}
in Figure 1a) is defined by the eight original nodes
(\nodenum{1}..\nodenum{8}). The side nodes added for the 4-node 2D
transformation (\nodenum{4}..\nodenum{8}) are defined by the two nodes
on the side. For example, the coordinates of node \nodenum{5} are the
average of the coordinates of nodes \nodenum{1} and \nodenum{2}. The
side nodes of a 20-node 3D element (\nodenum{21}..\nodenum{26} in Figure
1b) are defined by the eight nodes on the side. For example, the
coordinates of node \nodenum{21} are the average of the coordinates of
nodes \nodenum{1}, \nodenum{12}, \nodenum{4}, \nodenum{20}, \nodenum{8},
\nodenum{16}, \nodenum{5}, and \nodenum{17}. The center node
(\nodenum{27}) is defined by the original twenty nodes
(\nodenum{1}..\nodenum{20}). 

The \caps{ABAQUS} output file contains the nodal variables for
the original nodes.
The variables at the
generated nodes are the average of the variables at the original nodes
that define the new nodes as described for the coordinates.

\section{Translation of Element Quantities} \label{intro:element}

Each element must be assigned to an ``element block''. An element block
distinguishes a material or an element type (such as a truss or
quadrilateral). The \caps{ABAQUS} output file does not supply the
element block information, but it does contain an element type name for
each element with the element connectivity. The element block is either
determined by the element type name or supplied by the user (with the
\cmd{MATERIAL} command). If the element type is used, each type forms a
single block. The number of nodes per element must be the same for all
elements in an element block. 

In addition to the element block, the following information is needed
for each element: the dimension of the element, the element
connectivity, and the values of the element variables. 

The \caps{ABAQUS} output file does not supply the dimension of an
element, but it does contain the  number of nodes and an element type
name for each element with the element connectivity. Elements with less
than 4 nodes are assumed to be \caps{1D}, elements with 4 to 7 nodes are
\caps{2D}, and elements with more than 8 nodes are \caps{3D}. The
program distinguishes between 8-node \caps{2D} and \caps{3D} element
types by examining the element type name: it is \caps{3D} if and only if
the name contains the characters ``\cmd{3D}''. 

The \caps{ABAQUS} output file contains the connectivity for
the original elements.
Any element transformations are performed as explained in 
Section~\ref{intro:trans}.
The new elements replace the original
elements. 

The \caps{ABAQUS} output file contains the element variables 
for the original elements at appropriate integration points.
The variables for the
generated elements are read from the input variables at the appropriate
integration points. The \caps{ABAQUS} Manual has a diagram of the
positions of the integration points for each element type. If a
non-transformed element has more than one integration point, the
variables for the element are averaged over all the points.
