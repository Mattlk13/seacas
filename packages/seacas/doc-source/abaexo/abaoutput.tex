\chapter{Output} \label{chap:output}
%
%%% \errmsg produces an error message with explanation
\newcommand{\errmsg}[2]{
   {\sf #1}
   \topicbegin \negmedskip {#2} \topicend
   }

\caps{\PROGRAM} first scans the \caps{ABAQUS} file for initial
information such as the number of nodes and the variable records.
The \caps{ABAQUS} database is assumed to be correct and complete.
The
program may flag some \caps{ABAQUS} database errors at this point.
For example, there may be too many element variables selected, causing
some to be ignored.

\errmsg{WARNING - OVER \param{max} ELEMENT VARIABLES SELECTED,
record type \param{type} ignored \\
WARNING - OVER \param{max} NODAL VARIABLES SELECTED, record type
\param{type} ignored}
{
The maximum number of variables to be stored is exceeded. All following
record types are ignored. Commands \cmd{EVARS} and \cmd{NVARS} limit the
variable records selected. This message appears during the first pass.
}

After the \caps{ABAQUS} database is scanned, command input is requested
from the user. An error or warning message may appear in response to a
command. If an error message appears, the command is usually ignored. If
only a warning is printed, the command is usually performed. If the
message is not sufficiently informative, the appropriate command
description may be helpful.

The database is then rewound and the element connectivity and the nodal
coordinates are read. The new elements and nodes are generated and
written to the new \caps{EXODUS} database. \caps{\PROGRAM} displays
information about the input database and the generated elements, etc.
The element and nodal variables are then read, and the values for each
time step are written to the \caps{EXODUS} database. \caps{\PROGRAM}
displays the number of time steps written.

\errmsg{WARNING - END OF FILE in ELEMENT CONNECTIVITY \\
WARNING - END OF FILE in NODE COORDINATES \\
WARNING - END OF FILE in ELEMENT VARIABLES \\
WARNING - END OF FILE in NODAL VARIABLES \\
WARNING - END OF FILE before TIME STEPS}
{
An end of file was found while reading the listed records.
}

\errmsg{WARNING - UNKNOWN RECORD \param{type}}
{
An unknown record type was found. The record is ignored. This message
does not appear after the first time step.
}

\errmsg{WARNING - UNKNOWN ELEMENT TYPE \param{type} for element
\param{elem},
assume \param{2or3}D}
{
The listed element type in the connectivity record (1900) is unknown to
\caps{\PROGRAM}. The element is assumed to have the listed dimension and
is handled accordingly. Please contact the code sponsor.
}

\errmsg{WARNING - ELEMENT TYPE \param{type} NOT HANDLED for element
\param{elem},
assume \param{2or3}D}
{
The listed element type in the connectivity record (1900) is not handled
by \caps{\PROGRAM}. The element is assumed to have the listed dimension
and is handled accordingly.
}

\errmsg{WARNING - \# NODES for element \param{elem}, \param{2or3}D,
read \param{n}, expected \param{nexp}}
{
The number of nodes in the connectivity record (1900) is incorrect. For
3D elements, 8 or 20 nodes are expected; for 2D elements, 4 or 8 nodes
are expected. \caps{\PROGRAM} ignores extra nodes if the element is
transformed; otherwise they are kept.
}

\errmsg{WARNING - \# ELEMENTS in connectivity, read \param{n}, expected
\param{nexp}}
{
There are no connectivity records for some elements.
}

\errmsg{WARNING - \# NODES in coordinates, read \param{n}, expected
\param{nexp}}
{
Either there are two coordinate records (1901) for some nodes or there
are no coordinate records for some nodes.
}

\errmsg{WARNING - MISSING POINTS for element \param{elem},
variables \param{n$_1$} to \param{n$_2$}, \param{npt} \verb|<|
\param{exppt}}
{
The number of integration points expected is dependent on the element
transformation. There should be at least one integration point for each
new element. This message appears on the first time step only.
}

\errmsg{WARNING - \# ELEMENTS in step \param{s}, read \param{n},
expected \param{nexp}}
{
The number of elements in the variable records is not equal to the
maximum from the connectivity records.
}

\errmsg{WARNING - \# ELEMENT VARIABLES in step \param{s},
read \param{n}, expected \param{nexp}}
{
The number of element variables stored is not equal to the maximum
scanned from all the time steps.
}

\errmsg{WARNING - \# ELEMENT POINTS in step \param{s},
read \param{n}, expected \param{nexp}}
{
The number of integration points expected is dependent on the element
transformation. There should be at least one integration point for each
new element.
}

\errmsg{WARNING - \# NODES in step \param{s}, read \param{n}, expected
\param{nexp}}
{
The number of nodes in the variable records (51..1000) is not equal to
the maximum from the coordinate records.
}

\errmsg{WARNING - \# NODAL VARIABLES in step \param{s},
read \param{n}, expected \param{nexp}}
{
The number of nodal variables stored is not equal to the maximum scanned
from all the time steps.
}

\errmsg{WARNING - MISSING CONNECTIVITY for element \param{elem}, filled
with 0}
{
No connectivity record was read for the element.
}

The program allocates memory dynamically as it is needed. If the system
runs out of memory, the following message is printed:
\errfmt{
\cmd{FATAL ERROR - Too much dynamic memory requested}
}
and the program aborts. The user should first try to obtain more memory
on the system. Another solution is to run the program in a less
memory-intensive fashion. For example, restricting the variable records
selected should require less memory.

\caps{\PROGRAM} has certain programmer-defined limitations. The limits
are not specified in this manual since they may change. In most cases the
limits are chosen to be more than adequate. If the user exceeds a limit,
a message is printed. If the user feels the limit is too restrictive,
the code sponsor should be notified so the limit may be raised in future
releases of \caps{\PROGRAM}.
